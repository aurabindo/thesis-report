

\chapter{Introduction}
\label{report}
\paragraph*{•}

\hspace{8mm} 

\noindent Here goes your introduction. Its just a matter of choice if you want to have
that \textit{noindent} at the beginning.

Here starts another big paragraph.

\section{A section}

Here is more stuff.
\subsection {Hardware}

This is the first paragraph in the subsection. \\

Here is a subsection. You can more levels of sub sections. Notice how it
automatically starts indending the first line when a new paragraph is started,
but not the very first line in the section or a subsection ?
In the source file, you can have the content split across multiple lines. \\

\noindent One empty line stops a paragraph. But for an extra line,
you need to specifiy it explicitly with a double forward slash.\\

\noindent Now, to insert a figure:

\begin{figure}[h]
  \centering
    \centering
    \includegraphics[scale=0.3]{manipal.jpg}
    \caption{The Manipal University Logo}
    \label{fig:manipal_logo}
\end{figure}



%\begin{figure}[h]
%  \centering
%  \begin{subfigure}[b]{1\textwidth}
%    \centering
%    \includegraphics[scale=0.4]{digraph_simple.png}
%    \caption{relation digraph of the simplified system}
%    \label{fig:digraph_simple}
%  \end{subfigure}
  
%  \begin{subfigure}[b]{1\textwidth}
%    \centering
%    \includegraphics[scale=0.4]{digraph_virtual.png}
%    \caption{Representation of the digraph with virtual nodes}
%    \label{fig:digraph_virtual}
%  \end{subfigure}
  
%\end{figure}

